\documentclass{article}
\usepackage{graphicx} % Required for inserting images

\title{ODE Practice Problems}
\author{Swesan Pathmanathan}
\date{December 2025}

\begin{document}

\maketitle

\section*{Problems}

\subsection*{Problem 1 — First-Order ODE (Analytic Solution)}
Solve the initial value problem:
\[
y' = -y + t, \qquad y(0) = 5.
\]
Find the explicit closed-form solution \(y(t)\).

\subsection*{Problem 2 — Pendulum System Conversion}
The motion of a pendulum of length \(r = 1\) satisfies:
\[
\theta'' = -g \sin \theta, \qquad g = 9.81.
\]
Convert this second-order ODE into an equivalent first-order system in variables \(y_1, y_2\).  
State the required initial conditions.

\subsection*{Problem 3 — Forward Euler (Numeric Computation)}
Apply Forward Euler with step size \(h = 0.1\) to the IVP:
\[
y' = -y + t, \qquad y(0)=5.
\]
Compute \(y_1, y_2, y_3\).

\subsection*{Problem 4 — Backward Euler (Implicit Step)}
For the same IVP:
\[
y' = -y + t, \qquad y(0)=5,
\]
derive the Backward Euler iteration formula for \(y_{i+1}\), then compute  
\(y_1, y_2, y_3\) for \(h = 0.1\).

\subsection*{Problem 5 — Forward Euler Stability Condition}
Consider the linear test equation:
\[
y' = \lambda y,\qquad \lambda < 0.
\]
(a) Derive the stability condition for the Forward Euler method.  
(b) State the maximum allowable step size \(h\) that ensures numerical stability.

\subsection*{Problem 6 — Forward Euler Instability Demonstration}
Consider:
\[
y' = -10y,\qquad y(0)=1.
\]
(a) Using Forward Euler, write the recurrence relation.  
(b) Determine the stability bound for \(h\).  
(c) For \(h = 0.21\), compute the first three steps \(y_1, y_2, y_3\).  
(d) Explain why this step size leads to unstable growth or oscillation.

\subsection*{Problem 7 — Backward Euler Stability}
Apply Backward Euler to the linear test equation:
\[
y' = \lambda y,\qquad \lambda < 0.
\]
(a) Derive the Backward Euler update formula.  
(b) Show that the method is unconditionally stable (stable for all \(h>0\)).

\subsection*{Problem 8 — Backward Euler Computation}
For:
\[
y' = -10y,\qquad y(0)=1,
\]
compute the Backward Euler approximations \(y_1, y_2, y_3\) for  
(a) \(h = 0.05\),  
(b) \(h = 0.15\),  
(c) \(h = 0.50\),  
(d) \(h = 1.0\).

% ----------------------------
% WEEK 10 TUTORIAL PROBLEMS
% ----------------------------

\subsection*{Problem 9 — Van der Pol System Conversion}
Write the Van der Pol equation
\[
y'' - \mu(1 - y^2)y' + y = 0
\]
as a first-order system of ODEs.

\subsection*{Problem 10 — Effect of Round-Off Error on IVPs}
Consider the IVP:
\[
x' = x,\qquad x(0)=c.
\]
Suppose the value of \(c\) is perturbed when read into the computer as \(\tilde{c}=c+\varepsilon\).  
(a) Determine the error at \(t=10\) and \(t=20\).  
(b) Repeat the analysis for the IVP:
\[
x' = -x.
\]

\subsection*{Problem 11 — Trapezoidal Method on a Nonlinear ODE}
Consider the nonlinear ODE:
\[
y' = y^2.
\]
(a) Write the implicit trapezoidal method for this ODE.  
(b) Write the explicit trapezoidal (predictor–corrector) method.  
(c) Derive the nonlinear equation that must be solved for \(y_{i+1}\) in the implicit method.

% ----------------------------
% ASSIGNMENT ODE PROBLEMS (FILTERED & SIMPLIFIED)
% ----------------------------

\subsection*{Problem 12 — Stiff Linear System (Theory + Analysis)}
Consider the stiff linear ODE system:
\[
y' =
\begin{pmatrix}
-0.1 & -199.9 \\
0 & -200
\end{pmatrix} y,
\qquad y(0) = \begin{pmatrix}2 \\ 1\end{pmatrix}.
\]
The exact solution is:
\[
y_1(t) = e^{-0.1 t} + e^{-200 t}, \qquad 
y_2(t) = e^{-200 t}.
\]

(a) Compute the values of \(y_1(0.1)\), \(y_1(5)\), and \(y_1(100)\).  
(b) Explain why this system is considered \emph{stiff}.  
(c) Determine the dominant time scale and the fast time scale in the solution.  
(d) Explain why an explicit Euler method would require a extremely small stepsize to remain stable.  
(e) Briefly describe why implicit solvers (e.g., backward Euler) handle this problem efficiently.

% ----------------------------
% EXAM-STYLE RK4 PROBLEM
% ----------------------------

\subsection*{Problem 13 — Runge–Kutta 4th Order (One Step)}
Consider the initial value problem
\[
y' = -2y + t, \qquad y(0) = 1.
\]
Use one step of the classical 4th-order Runge–Kutta method (RK4) with stepsize
\[
h = 0.1
\]
to approximate \(y(0.1)\).

Write out the values of \(k_1, k_2, k_3, k_4\) and the final approximation \(y_1\).


\section*{Solutions}

\subsection*{Solution 1}
Solve:
\[
y'=-y+t.
\]
Homogeneous: \(y_h = Ce^{-t}\).  
Particular: \(y_p=t-1\).  
General solution:
\[
y(t)=t-1+Ce^{-t}.
\]
Use \(y(0)=5\):
\[
-1 + C = 5 \Rightarrow C = 6.
\]
\[
\boxed{y(t)=t-1+6e^{-t}}
\]

\subsection*{Solution 2}
Let:
\[
y_1=\theta,\quad y_2=\theta'.
\]
Then:
\[
y_1' = y_2, \qquad y_2' = -g\sin(y_1).
\]
Required initial conditions:
\[
y_1(0)=\theta(0),\quad y_2(0)=\theta'(0).
\]

\subsection*{Solution 3}
Forward Euler:
\[
y_{i+1} = y_i + h(-y_i + t_i), \quad h=0.1.
\]
\[
\begin{aligned}
y_1 &= 5 + 0.1(-5+0) = 4.5,\\
y_2 &= 4.5 + 0.1(-4.5+0.1)=4.06,\\
y_3 &= 4.06 + 0.1(-4.06+0.2)=3.674.
\end{aligned}
\]

\subsection*{Solution 4}
Backward Euler:
\[
y_{i+1} = y_i + h(-y_{i+1}+t_{i+1}).
\]
Solve for \(y_{i+1}\):
\[
y_{i+1}=\frac{y_i+ht_{i+1}}{1+h}.
\]
With \(h=0.1\):
\[
\begin{aligned}
y_1 &= \frac{5 + 0.1(0.1)}{1.1}=4.5545,\\
y_2 &= \frac{4.5545+0.1(0.2)}{1.1}=4.1586,\\
y_3 &= \frac{4.1586+0.1(0.3)}{1.1}=3.7987.
\end{aligned}
\]

\subsection*{Solution 5}
Forward Euler on \(y'=\lambda y\):
\[
y_{i+1} = (1+h\lambda)y_i.
\]
Stability requires:
\[
|1+h\lambda|\le 1.
\]
For \(\lambda < 0\):
\[
0<h\le \frac{2}{|\lambda|}.
\]

\subsection*{Solution 6}
Recurrence:
\[
y_{i+1} = (1 - 10h)y_i.
\]
Stability:
\[
|1-10h|\le 1 \Rightarrow h\le 0.2.
\]
For \(h=0.21\):
\[
\begin{aligned}
y_1 &= -1.1,\\
y_2 &= 1.21,\\
y_3 &= -1.331.
\end{aligned}
\]
Oscillation and growth → unstable.

\subsection*{Solution 7}
Backward Euler:
\[
y_{i+1}=y_i + h\lambda y_{i+1}.
\]
Solve:
\[
y_{i+1} = \frac{1}{1-h\lambda}y_i.
\]
Since \(\lambda<0\) and \(1-h\lambda>1\):
\[
\left|\frac{1}{1-h\lambda}\right|\le 1.
\]
Unconditionally stable for all \(h>0\).

\subsection*{Solution 8}
Backward Euler:
\[
y_{i+1}=\frac{1}{1+10h}y_i.
\]

(a) \(h=0.05\):
\[
y_1= \frac{1}{1.5},\quad
y_2= \frac{1}{1.5^2},\quad
y_3= \frac{1}{1.5^3}.
\]

(b) \(h=0.15\):
\[
y_i = (1/2.5)^i.
\]

(c) \(h=0.50\):
\[
y_i = (1/6)^i.
\]

(d) \(h=1.0\):
\[
y_i = (1/11)^i.
\]

% ----------------------------
% WEEK 10 TUTORIAL SOLUTIONS
% ----------------------------

\subsection*{Solution 9}
Let:
\[
x_1 = y,\qquad x_2 = y'.
\]
Then the system is:
\[
x_1' = x_2,\qquad
x_2' = \mu(1 - x_1^2)x_2 - x_1.
\]

\subsection*{Solution 10}
For \(x'=x\):
\[
x(t)=ce^t,\qquad \tilde{x}(t)=(c+\varepsilon)e^t.
\]
Error:
\[
e(t) = \tilde{x}(t)-x(t)=\varepsilon e^t.
\]
Thus:
\[
e(10)=\varepsilon e^{10},\qquad e(20)=\varepsilon e^{20}.
\]

For \(x'=-x\):
\[
x(t)=ce^{-t},\qquad \tilde{x}(t)=(c+\varepsilon)e^{-t}.
\]
Error:
\[
e(t)=\varepsilon e^{-t}.
\]
Thus:
\[
e(10)=\varepsilon e^{-10},\qquad e(20)=\varepsilon e^{-20}.
\]

\subsection*{Solution 11}
Since \(f(t,y)=y^2\):

(a) Implicit trapezoidal:
\[
y_{i+1}
= y_i + \frac{h}{2}(y_i^2 + y_{i+1}^2).
\]

(b) Explicit trapezoidal (predictor–corrector):
\[
\tilde{y}_{i+1} = y_i + h y_i^2,
\]
\[
y_{i+1}
= y_i + \frac{h}{2}(y_i^2 + \tilde{y}_{i+1}^2).
\]

(c) Nonlinear equation to solve:
\[
g(y_{i+1}) =
-y_{i+1}
+ y_i
+ \frac{h}{2}(y_i^2 + y_{i+1}^2)
= 0.
\]

\subsection*{Solution 12}

The system has exact solution
\[
y_1(t) = e^{-0.1t} + e^{-200t}, \qquad y_2(t) = e^{-200t}.
\]

(a) Compute values:

\[
\begin{aligned}
y_1(0.1) &= e^{-0.01} + e^{-20} \approx 0.99005 + 2.06\times 10^{-9}, \\
y_1(5)   &= e^{-0.5} + e^{-1000} \approx 0.60653 + 0, \\
y_1(100) &= e^{-10} + e^{-20000} \approx 4.54\times 10^{-5}.
\end{aligned}
\]

Thus the fast term disappears almost immediately.

(b) The system is stiff because it contains two very different decay rates:
\[
\lambda_1 = -0.1, \qquad \lambda_2 = -200.
\]
One mode decays slowly (timescale \(10\)), the other extremely fast (timescale \(0.005\)).  
Stiffness occurs when a solver must track both simultaneously.

(c) Time scales:

- Slow timescale:  
\[
\tau_{\text{slow}} = \frac{1}{0.1} = 10.
\]
- Fast timescale:  
\[
\tau_{\text{fast}} = \frac{1}{200} = 0.005.
\]

(d) Explicit Euler stability requires
\[
|1 + h\lambda| \le 1.
\]
For \(\lambda = -200\):
\[
h \le \frac{2}{200} = 0.01.
\]
Thus Euler would need step sizes **smaller than 0.01**, making it extremely inefficient.

(e) Implicit solvers (Backward Euler, BDF) are stable for any \(h > 0\) on negative \(\lambda\).  
Thus they take large steps while maintaining stability, making them appropriate for stiff systems.

\subsection*{Solution 13}

We apply RK4 with \(h = 0.1\) to the ODE \(y' = -2y + t\).

\[
f(t,y) = -2y + t.
\]

\textbf{Step 1: Compute the slopes}

\[
k_1 = f(0, 1) = -2(1) + 0 = -2.
\]

\[
k_2 = f\left(0 + \frac{h}{2},\, 1 + \frac{h}{2}k_1\right)
= f(0.05,\, 1 + 0.05(-2))
= f(0.05,\, 0.9).
\]
\[
k_2 = -2(0.9) + 0.05 = -1.75.
\]

\[
k_3 = f\left(0 + \frac{h}{2},\, 1 + \frac{h}{2}k_2\right)
= f(0.05,\, 1 + 0.05(-1.75))
= f(0.05,\, 0.9125).
\]
\[
k_3 = -2(0.9125) + 0.05 = -1.775.
\]

\[
k_4 = f(0.1,\, 1 + h k_3)
= f(0.1,\, 1 + 0.1(-1.775))
= f(0.1,\, 0.8225).
\]
\[
k_4 = -2(0.8225) + 0.1 = -1.545.
\]

\textbf{Step 2: Combine to obtain \(y_1\)}
\[
y_1 = 1 + \frac{h}{6}
\left(k_1 + 2k_2 + 2k_3 + k_4\right)
\]
\[
= 1 + \frac{0.1}{6}\left(-2 + 2(-1.75) + 2(-1.775) + (-1.545)\right).
\]

Compute the sum:
\[
-2 - 3.5 - 3.55 - 1.545 = -10.595.
\]

Thus:
\[
y_1 = 1 + \frac{0.1}{6}(-10.595)
= 1 - 0.17658
= 0.82342.
\]

\[
\boxed{y(0.1) \approx 0.8234}
\]


\end{document}
